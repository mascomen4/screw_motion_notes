\documentclass[12pt]{article}

\usepackage[no-math]{fontspec}
\usepackage{polyglossia}
\setdefaultlanguage[
  indentfirst=true,
  spelling=modern]{russian}
\setotherlanguage{english}

\usepackage{amsmath}
\usepackage{unicode-math}

\setmainfont{CMU Serif}
\setsansfont{CMU Sans Serif}
\setmonofont{CMU Typewriter Text}

\setmathfont{Latin Modern Math}

\usepackage{cancel}
% \usepackage{latexsym}
\usepackage{hyperref} % ссылки в документе
\hypersetup{
  colorlinks,
  citecolor=black,
  filecolor=black,
  linkcolor=black,
  urlcolor=black,
  bookmarksdepth=subsubsection, 
  unicode=true,       
  pdftoolbar=true,       
  pdfmenubar=true,    
  pdffitwindow=false,  
  pdfstartview={FitH},  
  pdftitle={Название},
  pdfauthor={Коллектив авторов}
}
\usepackage{graphicx} % картинки
\usepackage{xcolor} % цвет


%~~~~~~~~~~~~~~~~~~~~~~~~~~~~~~~~~~~~~~~~~~~~~~~~~~~~~~~~~~~~~~~~~~~~~~~~~~~~~~~
%								BibLaTeX
%~~~~~~~~~~~~~~~~~~~~~~~~~~~~~~~~~~~~~~~~~~~~~~~~~~~~~~~~~~~~~~~~~~~~~~~~~~~~~~~
\usepackage[autostyle]{csquotes}
\usepackage{url}

\usepackage[%
  backend=biber,
  bibstyle=gost-numeric,
  defernumbers=true,
  movenames=false, % не менять местами заголовок и список авторов, если авторов больше четырех
  maxbibnames=10, % сколько авторов указывать
  sorting=none,
]{biblatex}
% Файл с литературой
\addbibresource{bib/algebra.bib}
\addbibresource{bib/zotero/zotero.bib}