%%--------------------------
\begin{frame}
  \frametitle{Элементарные вращения}
  Матрицы вращения вокруг осей декартовой системы координат:
  \begin{equation*}
    R_{x}(\alpha) = 
    \begin{bmatrix}
      1 & 0 & 0\\
      0 & \cos\alpha & -\sin\alpha\\
      0 & \sin\alpha & \cos\alpha
    \end{bmatrix}
    \;\;
    R_{y}(\beta) = 
    \begin{bmatrix}
      \cos\beta & 0 & \sin\beta \\
      0 & 1 & 0\\
      -\sin\beta & 0 & \cos\beta
    \end{bmatrix}
    \;\;
    R_{z}(\gamma) = 
    \begin{bmatrix}
      \cos\gamma & -\sin\gamma & 0\\
      \sin\gamma & \cos\gamma & 0\\
      0 & 0 & 1
    \end{bmatrix}
  \end{equation*}
  \begin{itemize}
    \item Вращается радиус вектор $\vb{p}$ или свободный вектор $\vb{v}$ вокруг оси $Ox$, $Oy$ или $Oz$.
    \item Вращение происходит против часовой стрелки в правой системе координат.
    \item Система координат остается неподвижной.
  \end{itemize}
  По теореме Эйлера любой поворот в трехмерном пространстве можно представить как последовательность из трех элементарных поворотов, где никакие два поворота подряд не осуществляются вокруг одной и той же оси. Это дает 12 разных комбинаций:
  \begin{equation*}
    \begin{matrix}
      XYZ & YZX & ZXY & XZY & YXZ & ZYX \\
      XYX & YZY & ZXZ & XZX & YXY & ZYZ
    \end{matrix}
  \end{equation*}
\end{frame}