\documentclass[
  8pt,
  aspectratio=169,
  russian,
  lualatex,
  usepdftitle={false},
  usenames,
  dvipsnames,
  % ignorenonframetext,
  % notheorems,
  % draft,
]{beamer}

  
% \documentclass[russian,12pt]{article}
% \usepackage[noamssymb]{beamerarticle}

\mode<article> {
  %% Задаем размер полей путем указания области, занимаемой текстом
  \usepackage[a4paper, total={6.5in, 10in}]{geometry}
  \usepackage{fullpage}
  % В режиме презентации hyperref загружается сам
  \usepackage{hyperref}
  \usepackage{graphicx}
}

%~~~~~~~~~~~~~~~~~~~~~~~~~~~~~~~~~~~~~~~~~~~~~~~~~~~~~~~~~~~~~~~~~~~~~~~~~~~~~~~
% Настройки, специфические для режима презентации
%~~~~~~~~~~~~~~~~~~~~~~~~~~~~~~~~~~~~~~~~~~~~~~~~~~~~~~~~~~~~~~~~~~~~~~~~~~~~~~~
\mode<presentation>{%
  \usetheme{metropolis}
  \metroset{%
    % показывать отдельный слайд  для заголовков/подзаголовков
    sectionpage=progressbar, %none, simple, progressbar
    subsectionpage=progressbar, 
    numbering=fraction, % номер слайда/ все слайды
    progressbar=foot, % прогресс всех слайдов
    block=fill
  }
  % Делаем фон слайдов белого цвета, а шрифт черного
  \setbeamercolor{background canvas}{bg=white}
  \setbeamercolor{normal text}{fg=black}
  % Заголовок слайдов, фоновый цвет
  \setbeamercolor{frametitle}{bg=Blue}
  \setbeamercolor{alerted text}{fg=Red}
  % Размер слайда
  \setbeamersize{text margin left=5mm,text margin right=5mm}
  % Нумеровать слайды при автоматическим разбиении
  \setbeamertemplate{frametitle continuation}{\insertcontinuationcount}
  % Не уменьшать размер шрифта для подпунктов пунктов списка
  % to set the body size
  \setbeamerfont{itemize/enumerate subbody}{size=\normalsize}
  % Символ для маркера списка второго уровня
  % \setbeamertemplate{itemize subitem}{\normalsize\hbox{\donotcoloroutermaths$\blacktriangleright$}}
  % Настройка блоков
  \setbeamertemplate{blocks}[rounded,shadow=false]
  \setbeamercolor{block title}{bg=gray!10,fg=black}
  \setbeamercolor{block body}{bg=gray!10,fg=black}
  % Блок example
  \setbeamercolor{block title example}{bg=gray!10,fg=black}
  \setbeamercolor{block body example}{bg=gray!10,fg=black}
  % Блок theorem
  \setbeamercolor{block title theorem}{bg=gray!10,fg=black}
  \setbeamercolor{block body theorem}{bg=white,fg=black}
}
%~~~~~~~~~~~~~~~~~~~~~~~~~~~~~~~~~~~~~~~~~~~~~~~~~~~~~~~~~~~~~~~~~~~~~~~~~~~~~~~
%              Настройка шрифтов
%~~~~~~~~~~~~~~~~~~~~~~~~~~~~~~~~~~~~~~~~~~~~~~~~~~~~~~~~~~~~~~~~~~~~~~~~~~~~~~~
\usepackage[no-math]{fontspec}
% При использовании LuaLaTeX отключаем следующие два пакета
% \usepackage{xunicode}
% \usepackage{xltxtra}

\usepackage{polyglossia}
\setdefaultlanguage[indentfirst=true,spelling=modern]{russian}
\setotherlanguage{english}
\setotherlanguage{greek}

\usepackage{tabularray} % Пакет для таблиц
\UseTblrLibrary{amsmath}
% \usepackage{amsmath}
\usepackage{unicode-math}
\usepackage{accents}

\unimathsetup{
  % math-style=french,
  math-style=TeX,
  % bold-style=ISO
  mathbf=sym,
  mathrm=sym,
}

\setmainfont[Ligatures=TeX]{IBM Plex Serif}
\setromanfont[Ligatures=TeX]{IBM Plex Serif}
\setsansfont[Ligatures=TeX]{IBM Plex Sans}
\setmonofont[Ligatures=TeX]{IBM Plex Mono}

% \setmainfont[Ligatures=TeX]{PT Serif}
% \setsansfont[Ligatures=TeX]{PT Sans}
% \setmonofont[Ligatures=TeX]{PT Mono}

% \newfontfamily\greekfontsf{Alegreya Sans}

% \setmainfont[Ligatures=TeX]{CMU Serif}
% \setsansfont[Ligatures=TeX]{CMU Sans Serif}
% \setmonofont{CMU Typewriter Text}

% \setmathfont{Latin Modern Math}
% \setmathfont{TeX Gyre Bonum Math}
\setmathfont{TeX Gyre Pagella Math}
% \setmathfont{TeX Gyre Schola Math}
% \setmathfont{TeX Gyre Termes Math}
% \setmainfont{Libertinus Math}
% \setmainfont{Cambria Math}
% \setmathfont{XITS Math}
% \setmathfont{STIX}
% \setmathfont{DejaVu Math TeX Gyre} % нет
% \setmathfont{Asana-Math fonts} % нет
% \setmainfont{Lucida Bright Math} % нет
% \setmainfont{Fira Math} % нет
% \setmainfont{Minion Math} % нет
%
%~~~~~~~~~~~~~~~~~~~~~~~~~~~~~~~~~~~~~~~~~~~~~~~~~~~~~~~~~~~~~~~~~~~~~~~~~~~~~~~
%								Пакеты для математики
%~~~~~~~~~~~~~~~~~~~~~~~~~~~~~~~~~~~~~~~~~~~~~~~~~~~~~~~~~~~~~~~~~~~~~~~~~~~~~~~
% \usepackage{latexsym} % Некоторые дополнительные символы (например стрелки)
% \usepackage{mathrsfs} % Рукописное начертание (с завитушками)
% \usepackage{amsthm} % Теоремы и т.п.
% \usepackage{amsopn} % Пакет для объявления новых математических операторов
% \usepackage{tensor}
% \usepackage{cancel}
\usepackage{physics}

%~~~~~~~~~~~~~~~~~~~~~~~~~~~~~~~~~~~~~~~~~~~~~~~~~~~~~~~~~~~~~~~~~~~~~~~~~~~~~~~
%				Настройка пакета hyperref
%~~~~~~~~~~~~~~~~~~~~~~~~~~~~~~~~~~~~~~~~~~~~~~~~~~~~~~~~~~~~~~~~~~~~~~~~~~~~~~~
\hypersetup{
    unicode=true,          % non-Latin characters in Acrobat’s bookmarks
    pdftoolbar=true,        % show Acrobat’s toolbar?
    pdfmenubar=true,        % show Acrobat’s menu?
    pdffitwindow=false,     % window fit to page when opened
    pdfstartview={FitH},    % fits the width of the page to the window
    bookmarksdepth=subsection,
    colorlinks=true,
    linkcolor={blue},
    pdftitle={Винты и бикватернионы},
    pdfauthor={Геворкян М. Н.}
}
%~~~~~~~~~~~~~~~~~~~~~~~~~~~~~~~~~~~~~~~~~~~~~~~~~~~~~~~~~~~~~~~~~~~~~~~~~~~~~~~
%								Разные пакеты
%~~~~~~~~~~~~~~~~~~~~~~~~~~~~~~~~~~~~~~~~~~~~~~~~~~~~~~~~~~~~~~~~~~~~~~~~~~~~~~~
% \usepackage{minted} % Пакет для листингов
% \usepackage{wrapfig}
% \usepackage{multirow}
% \usepackage{pgf}
% \usepackage{tikz}
% \usetikzlibrary{tikzmark}
% \usetikzlibrary{arrows}
%\usepackage{verbatim}% Для исходного кода, цитат и т.д. и т.п.
%\usepackage{cite}

% Дополнительные значки для маркеров списка
% \usepackage{pifont}

% Считывание csv файлов и оформление в виде таблиц
% \usepackage{pgfplotstable}
% \pgfplotsset{compat=newest}
% \pgfplotstableset{
%   every head row/.style={after row=\hline},
% }
% pgfplotstable рекомендуется использовать вместе с
% тремя нижеподключенными пакетами:
% \usepackage{booktabs}
% \usepackage{array}
% \usepackage{colortbl}
%~~~~~~~~~~~~~~~~~~~~~~~~~~~

%~~~~~~~~~~~~~~~~~~~~~~~~~~~~~~~~~~~~~~~~~~~~~~~~~~~~~~~~~~~~~~~~~~~~~~~~~~~~~~~
%								BibLaTeX
%~~~~~~~~~~~~~~~~~~~~~~~~~~~~~~~~~~~~~~~~~~~~~~~~~~~~~~~~~~~~~~~~~~~~~~~~~~~~~~~
\setbeamertemplate{bibliography item}{\insertbiblabel}
\usepackage[autostyle]{csquotes}
\usepackage{url}

\usepackage[%
  backend=biber,
  hyperref=auto,
  language=auto,
  autolang=other,
  otherlangs=true,
  langhook=extras,
  style=gost-numeric,
  citestyle=gost-numeric,
  bibstyle=gost-numeric,
  defernumbers=true,
  refsection=section,
  movenames=false,% не менять местами заголовок и список авторов, если авторов больше четырех
  maxbibnames=3,% сколько авторов указывать
  % sorting=ydnt
]{biblatex}
% Файлы с литературой, берем из монографии
\newcommand{\bibsdir}{../../Монография/bib}
% \addbibresource[glob=true]{\bibsdir/*.bib}
% Прописываем явно, иначе не работают подсказки в VS Code
\addbibresource{\bibsdir/algebra.bib}
\addbibresource{\bibsdir/animation.bib}
\addbibresource{\bibsdir/asymptote.bib}
\addbibresource{\bibsdir/biquaternions.bib}
\addbibresource{\bibsdir/computer_graphics.bib}
\addbibresource{\bibsdir/computer_geometry.bib}
\addbibresource{\bibsdir/differential_geometry.bib}
\addbibresource{\bibsdir/geometry.bib}
\addbibresource{\bibsdir/kinematics.bib}
\addbibresource{\bibsdir/physics.bib}
\addbibresource{\bibsdir/projective_geometry.bib}
\addbibresource{\bibsdir/python.bib}
\addbibresource{\bibsdir/quaternions.bib}
\addbibresource{\bibsdir/rotations.bib}
\addbibresource{\bibsdir/soft.bib}
\addbibresource{\bibsdir/splines.bib}
\addbibresource{\bibsdir/history.bib}

%~~~~~~~~~~~~~~~~~~~~~~~~~~~~~~~~~~~~~~~~~~~~~~~~~~~~~~~~~~~~~~~~~~~~~~~~~
%					Определение новых команд
%~~~~~~~~~~~~~~~~~~~~~~~~~~~~~~~~~~~~~~~~~~~~~~~~~~~~~~~~~~~~~~~~~~~~~~~~~
\newtheorem{statement}[theorem]{Утверждение}
% Переопределим \emph чтобы вместо курсива был красный шрифт
\renewcommand<>{\emph}[1]{%
  {\usebeamercolor[Red]{emph}\only#2#1}%
}
\renewcommand{\Im}{\mathop{\mathrm{Im}}\nolimits} % мнимая часть числа
\renewcommand{\Re}{\mathop{\mathrm{Re}}\nolimits} % действительная часть числа
\newcommand{\const}{\mathrm{const}} % константа

\newcommand{\noteq}{\stackrel{\mathrm{not}}{=}} % обозначение
\newcommand{\defeq}{\stackrel{\mathrm{def}}{=}}	% определение

\renewcommand{\i}{\mathrm{i}}
\renewcommand{\j}{\mathrm{j}}
\renewcommand{\k}{\mathrm{k}}